% DO NOT EDIT--COPIED FROM NETWORK_SURVEY
\begin{description}

\item[Authorship networks] are unweighted bipartite networks
consisting of links between authors and their works.  In some authorship
networks such as that of scientific literature, works have typically a few
authors, whereas works in other authorship networks may have many
authors, as in Wikipedia articles.  In an authorship network, vertices
corresponding to works are added one by one with all their incident
edges, and the link prediction problem is not meaningful.  Therefore, we
also consider author-author networks of co-authorships.  In these
networks, each edge represents a joint authorship, with the timestamp
representing the publication date. 
Example of authorship networks are the Wikipedia user--article edit
network, and the DBLP network of scientists and their publications. 

\item[Communication networks] contain edges that represent
single messages between persons.  Communication networks are directed
and may have multiple edges.  Timestamps denote the date of a message.
Examples of communication networks include emails, Facebook wall posts
and Twitter posts addressed using the ``@name'' notation.

\item[Co-occurrence networks] represent the simultaneous appearance of
  items.  Co-occurrence networks are unipartite, undirected and
  unweighted.  Examples are the co-purchase network from Amazon, and the
  \emph{is-similar} relationship of DBpedia.

\item[Feature networks] are bipartite, and denote any kind of feature
assigned to entities.  This includes user memberships in online groups,
as well as player memberships in sports clubs.  Feature networks are
unweighted and have edges that are not annotated with edge creation
times. 
Examples are the genre of songs and the clubs in which a football player
has played. 

\item[Folksonomies] consist of tag assignments connecting a user, an
item and a tag.  For folksonomies, we follow the 3-bipartite 
approach and consider the three possible bipartite networks, i.e.\ the
user--item, user--tag and item--tag networks.  This allows us to apply
methods for bipartite graphs to hypergraphs, which is not possible
otherwise.  
Examples of folksonomies are Delicious for bookmarks, CiteULike for
scientific publications and MovieLens for movies. 

\item[Interaction networks] represent collections of single events
between entities.  Most interaction networks have edges with
timestamps. 
Examples are the interaction of proteins and other molecules in
biological organisms and the Last.fm user--song listening network. 

\item[Physical networks] represent physically existing network
structures.  This includes physical computer networks and road
networks.  
These networks result from an underlying two-dimensional geographical
layout. 
Examples are road networks and the autonomous systems of the Internet.  

\item[Rating networks] consist of assessments given to entities 
by users, weighted by a rating value.   Most rating networks are
bipartite, when users rate items.  A few rating networks are unipartite
and directed,
when users rate other users.  Most rating networks are weighted; others
are signed, when there is an explicit neutral rating value of zero. 
Examples are the Netflix movie ratings and Jester joke ratings. 

\item[Reference networks] consist of citations or
hyperlinks between publications, patents or web pages.   Reference networks are
directed. 
In most reference networks, edges are created along with nodes.
Therefore, the link prediction problem cannot be applied to them.  An
exception are hyperlink networks, where links can be added at any time. 
Examples are hyperlinks between pages on the World Wide Web and
citations between scientific publications. 

\item[Social networks] represent relations of friendship between
persons.  Some social networks have negative edges, which denote enmity.
All social networks have simple or signed edges, since it is not
possible to add another user multiple times to one's friend (or foe)
list. 
In social networks, timestamps denote when a link was established.  
Examples are Facebook friendships, the Twitter follower relationship,
and friends and foes on Slashdot. 

\item[Trust networks] connect persons or entities by links of
trust.  Trust networks are necessarily directed.  Links may also denote
distrust, in which case negative edge weights are used. 
Examples are given by the sites Epinions and Advogato. 

\end{description}
